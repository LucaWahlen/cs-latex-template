\phantomsection
\addcontentsline{toc}{section}{Sperrvermerk}
\section*{Sperrvermerk}

\markright{Sperrvermerk}

Ein Sperrvermerk ist nur zulässig, wenn \textbf{zwingend schützensnotwendige}, vertrauliche Informationen in der Arbeit enthalten sind. Dies ist normalerweise nur dann anzunehmen, wenn betriebsinterne Informationen des Arbeitsgebers \textbf{ohne Generalisierung, Abstrahierung oder Pseudonymisierung wiedergegeben werden}.

Die Überprüfung der Quellen durch die Gutachter muss trotz Sperrvermerk möglich sein. Die mit der Begutachtung und Prüfung von der Hochschule beauftragten Personen (direkt und indirekt) müssen zur Nutzung der Inhalte der Prüfungsleistung berechtigt sein.

Eine Nutzung der (digitalen) Dokumente für Zwecke der Plagiatsüberprüfung muss erlaubt sein.

Sperrvermerke, die hierzu in Widerspruch stehen, sind wahlweise nichtig oder führen dazu, dass die Arbeit als nicht abgegeben angesehen wird /die Prüfungsleistung als nicht erbracht bewertet wird.

In begründeten Fällen (siehe oben) kann ein Sperrvermerk wie folgt aussehen:

Diese Arbeit wurde unter Verwendung von vertraulichen Informationen der Firma „XXX -- bitte ersetzen" angefertigt. Diese Arbeit darf Dritten ohne ausdrückliche Zustimmung der Firma „XXX -- bitte ersetzen" nicht zugänglich gemacht werden.

Alle Nutzungen für die Begutachtung durch die Hochschule und von ihr beauftragte Dienstleister sind explizit erlaubt. Zu der Begutachtung gehört auch die Prüfung auf Plagiate und die Verwendung künstlicher Intelligenz unter Nutzung entsprechender IT-Plattformen.
