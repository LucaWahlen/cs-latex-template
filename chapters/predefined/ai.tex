\section{KI-Verzeichnis}
Jede Nutzung von KI ist zu dokumentieren. Dazu wird hinter dem Literaturverzeichnis ein separates KI-Verzeichnis eingefügt, das alle KI-generierten Inhalte, die eingesetzten Systeme, die verwendeten „Prompts" sowie die weitere Verwendung des Outputs der KI transparent macht.

\begin{tabular}{|p{0.10\textwidth}|p{0.60\textwidth}|p{0.30\textwidth}|}
    \hline
    \rowcolor{gray!30}
    \textbf{System} & \textbf{Prompt} & \textbf{Verwendung} \\
    \hline
    ChatGPT 1 & What criteria should I use to select a leader? & weiterentwickelt \\
    \hline
    ChatGPT 2 & Schreibe einen Text, in dem die folgenden Themen behandelt werden: Personalmarketing und seine Bedeutung für ein Unternehmen -- der Zusammenhang zum Employer Branding -- die Auswirkungen der Personalmarketingstrategie auf das Recruiting. & verändert: Passagen ausgelassen \\
    \hline
    ChatGPT 3 & Entwurf einer Gliederung für eine Hausarbeit zum Thema Recruiting & unverändert \\
    \hline
    Elicit 1 & Which elements should be included in an Employer Branding Plan? & Passagen überarbeitet \\
    \hline
\end{tabular}

\textit{Sofern keine KI verwendet wurde, enthält das Verzeichnis nur den Eintrag: „Es wurde keine KI verwendet."}
